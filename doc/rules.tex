\documentclass[11pt,a4paper]{article}
\XeTeXlinebreaklocale "zh"
\XeTeXlinebreakskip = 0pt plus 1pt minus 0.1pt
\usepackage[top=1in,bottom=1in,left=1.25in,right=1.25in]{geometry}
\usepackage{float}
\usepackage{fontspec}
\newfontfamily\zhfont[BoldFont=STHeiti]{STFangsong}
\newfontfamily\zhpunctfont{STFangsong}
\setmainfont{Times New Roman}
\usepackage{indentfirst}
\usepackage{zhspacing}
\zhspacing

\usepackage{amsmath}

\title{游戏规则说明}
\author{队式16开发组}

\begin{document}

  \maketitle

  \section{地图元素}
    地图采用\texttt{MapSize} * \texttt{MapSize}的矩阵棋盘式地图,方格有水下(下/0)、水面(中/1)、空中(上/2)三个层次。同一格子的同一层次同时只能被一个单位占据。

    地形:海洋、陆地

    \begin{description}
      \item[海洋] 相当于空地,上中下层分别可停留一个单位。
      \item[陆地] 建筑和资源点均在陆地地形上,陆地地形只有空中可停留单位。
    \end{description}


  \section{建筑}
    建筑包括基地、据点(相当于原来的岛屿)。基地和据点占据一矩形区域(不一定1×1方格)。

    \subsection{基地}
    \subsection{岛屿}



  \section{单位(可移动单位)}
    可移动单位:飞机、潜艇、船舰。

    可移动单位共有属性:

    \begin{description}
      \item[\texttt{Destination[2]}] 移动目的地
      \item[\texttt{Command}] 指令
      \item[\texttt{CommandCoordinate[3]}] 指令目标坐标
    \end{description}

    可移动单位占一个格子。

    \subsection{飞机}
    \subsection{潜艇}
    \subsection{船舰}




  \section{作战系统}
    \subsection{游戏模式}
      每回合双方玩家首先为所有可移动单位设定移动目的地\texttt{Destination},如果未设定,则按上回合的目的地继续移动。再设定本回合移动后的指令\texttt{Command}:攻击、补给、维修(每回合只能设定一条指令)。双方设定完所有指令后共同结算。

    \subsection{胜利条件}
      摧毁对方基地 或 回合数达到最大回合数\texttt{RoundMax}时,积分高的一方获胜。详见积分系统\ref{sec:score-system}。

    \subsection{距离}
      水平距离与竖直距离之和

    \subsection{初始条件}
      双方所有单位(包括基地)资源充满,满血。岛屿满血无主。

    \subsection{视野范围}
      地形全可见(海洋、基地、岛屿、油田、矿场),每个单位均有三层视野范围,己方单位共享视野范围,构成地图可见状态\texttt{FieldofVison[MapSize][MapSize][3]},-2表示不可见,-1表示无单位,否则为单位ID。

    \subsection{移动}
      移动时只能在同一层次内移动,且移动时不能穿越同层其他单位,同层单位亦不能堆叠于同一格。每移动一格消耗一单位燃料,燃料$\leq 0$则不能继续移动。特别地,若飞机编队燃料降至零,则坠毁,并对当前坐标的1层(水面)单位造成一定伤害。若飞机编队未发生实际移动(可能发出了移动指令但并未成功),则视为原地盘旋一回合,同样要消耗一单位燃料。

      \paragraph{特殊情况}
      当某两个单位在移动过程中将通过同一坐标时,则速度\texttt{Speed}大的单位移动成功并继续移动,移动未成功的单位停留在前一格不能继续移动,若两单位速度相同,则采取随机的方式决定哪个单位移动成功。

    \subsection{攻击}
      攻击在移动结束(包括上述意外停止的情况)后发动,攻击时需指定攻击的三维坐标点\texttt{CommandCoordinate[3]},即攻击地图某一坐标的下中上三层的某一层。如果该坐标不在该单位的相应视野范围内,则返回错误值,视为放弃此次攻击。弹药量$\leq 0$时不能发动攻击。弹药在回合结束时到达该坐标产生伤害(即对方可以通过移动回避攻击)。异层攻击不考虑弹药攻击路径;某些情况(具体哪些情况暂未确定)需考虑弹药的攻击路径,若攻击路径上有其他单位,则弹药会击中障碍并产生相应伤害(注意:有友军伤害!)。

    \subsection{补给}
      运输舰、航母、岛屿、基地特有指令,航母、岛屿、基地可以对任意单位补给,运输舰不能补给飞机。补给在移动结束后发动,同样需指定补给目标坐标,对空补给距离为0,对水面补给距离为1,对水下补给距离为0,1。超出范围则视为放弃补给。每次补给补满对方容量,若剩余量不足以补满对方,则全部补充给对方。(这种情况下,运输舰、航母是否需要自动保留一些燃料供自己使用?)

    \subsection{收集}
      运输舰特有指令,从岛屿或资源点获取资源,每次收集补满运输舰容量。收集距离为1。

    \subsection{维修}
      基地特有指令,对空维修距离为0,其余维修距离为1,一次只能维修一个单位,维修补满生命值,补满弹药及燃料。若被维修单位为飞机,则类似于生产操作,可选择维修后的编队配置方案。


  \section{资源与生产系统}
    资源点分为油田和矿场,均在陆地地形上。

    \paragraph{油田}
      不可攻击,油储备总量\texttt{TotalFuel},双方基地附近各有一小油田,地图的中间区域可设大油田,油田储备的燃料只能由运输舰收集获得(不可攻占,但可以派单位在周围保证运输舰运输安全)。

    \paragraph{矿产}
      不可攻击,金属储备总量\texttt{TotalMetal},双方基地附近各有一小矿场,地图的中间区域可设大矿场,矿场储备的金属只能由运输舰收集获得(不可攻占,但可以派单位在周围保证运输舰运输安全)。

    \paragraph{生产 Build}
      基地除了常规指令外,额外拥有生产指令,给出各单位对应的编号,若生产飞机编队,还需给出编队配置方案B\texttt{uildFormationNum[4]}。常量:生产飞机编队各机种所需资源及回合 \texttt{BuildFormation[4][4]}


  \section{积分系统}
  \label{sec:score-system}
    积分 = 对敌军造成的伤害 + 占领岛屿(每回合加分)+ 收集的资源量


  \section{伤害计算}


\end{document}
