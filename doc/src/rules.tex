\documentclass[11pt,a4paper]{article}
\XeTeXlinebreaklocale "zh"
\XeTeXlinebreakskip = 0pt plus 1pt minus 0.1pt
\usepackage[top=1in,bottom=1in,left=1.25in,right=1.25in]{geometry}
\usepackage{float}
\usepackage{fontspec}
\newfontfamily\zhfont[BoldFont=STHeiti]{STFangsong}
\newfontfamily\zhpunctfont{STFangsong}
\setmainfont{Times New Roman}
\setmonofont{DejaVu Sans Mono}
\usepackage{indentfirst}
\usepackage{zhspacing}
\zhspacing

\renewcommand{\descriptionlabel}[1]{\ttname{#1}}
\newcommand{\ttname}[1]{\mbox{ \ttfamily{#1}}}
\linespread{1.2}

\usepackage{amsmath}
\usepackage{enumerate}
\usepackage[colorlinks=true]{hyperref}

\begin{document}

  \title{游戏规则说明}
  \author{队式16开发组}
  \maketitle

  \section{地图元素}
    地图采用$\text{\ttfamily MapSize} * \text{\ttfamily MapSize}$的矩阵棋盘式地图,方格有水下(下~/~0)、水面(中~/~1)、空中(上~/~2)三个层次。同一格子的同一层次同时只能被一个单位占据。

    \paragraph{地形} 海洋、陆地
      \subparagraph{海洋} 相当于空地,上中下层分别可停留一个单位。
      \subparagraph{陆地} 建筑和资源点均在陆地地形上,陆地地形只有空中可停留单位。


  \section{建筑}
    建筑包括基地、据点(相当于原来的岛屿)。基地和据点占据一矩形区域(不一定$1×1$方格)。

    \subsection{基地}
      \paragraph{状态量}
        \begin{minipage}[t]{0.8\textwidth}
          \begin{description}
            \item[BaseHealth] 生命值
            \item[BaseFuel] 燃料
            \item[BaseMetal] 金属
          \end{description}
          弹药无限
        \end{minipage}

      \paragraph{常量}
        \begin{minipage}[t]{0.8\textwidth}
          \begin{description}
            \item[BaseHealthMax] 生命值上限
            \item[BaseFuelMax] 燃料上限
            \item[{BaseSigntRange[3]}] 视野范围(三层)
            \item[{BaseFireRange[3]}] 攻击范围
            \item[{BaseAttack[2]}] 攻击力(火力伤害和鱼雷伤害)
            \item[BaseMetalMax] 金属上限
          \end{description}
        \end{minipage}

      \paragraph{操作} 攻击、补给、维修。详见第\ref{sec:作战系统}节
      \paragraph{其他说明} 基地不可移动,\textbf{不可维修},即生命值无法恢复。


    \subsection{岛屿}
      \paragraph{状态量}
        \begin{minipage}[t]{0.8\textwidth}
          \begin{description}
            \item[IslandTeam] 所属阵营
            \item[IslandHealth] 生命值
            \item[IslandFuel] 燃料
            \item[IslandMetal] 金属
          \end{description}
        \end{minipage}

      \paragraph{常量}
        \begin{minipage}[t]{0.8\textwidth}
          \begin{description}
            \item[IslandHealthMax] 生命值上限
            \item[IslandFuelMax] 燃料上限
            \item[{IslandSigntRange[3]}] 视野范围(三层)
            \item[{IslandFireRange[3]}] 攻击范围
            \item[{IslandAttack[2]}] 攻击力(火力伤害和鱼雷伤害)
          \end{description}
        \end{minipage}

      \paragraph{操作} 攻击、补给。
      \paragraph{占领}
        游戏开始时,所有岛屿均为\textbf{无主状态},无主状态岛屿\textbf{无燃料和弹药储备},故\textbf{无攻击能力}。双方可以对岛屿发动攻击,当岛屿的生命值降为0时,岛屿的所有权归于\textbf{使岛屿生命力降为0的攻击发出方}。岛屿被占领后,占领方可通过运输舰向岛屿运输资源,也可将岛屿的资源运往其他地方。只有占领方可以控制岛屿进行攻击或补给。攻下对方的岛屿后,对方剩余的\textbf{资源不消失},转为己方岛屿的资源可任意利用。


  \section{单位(可移动单位)}
    \paragraph{可移动单位} 飞机、潜艇、船舰。

    \paragraph{可移动单位共有属性}
      \begin{minipage}[t]{0.6\textwidth}
        \begin{description}
          \item[{Destination[2]}] 移动目的地
          \item[Command] 指令
          \item[{CommandCoordinate[3]}] 指令目标坐标
        \end{description}
      \end{minipage}

    \paragraph{} 可移动单位占一个格子。


    \subsection{飞机}
      \paragraph{编队}
        地图上一个格子内的飞机并不是一架飞机,而是一个飞机编队,每个飞机编队飞机数为常量\ttname{FormationSize}。生产飞机时,玩家可自由安排编队内机种配置方案。编队内共四个机种:战斗机(0)、鱼雷机(1)、轰炸机(2)、侦察机(3)。不同的配置方案会对一个飞机编队(在地图上则呈现为一个飞机单位)的攻击、侦查范围、生命值产生影响。飞机编队不可合并。

      \paragraph{状态量}
        \begin{minipage}[t]{0.8\textwidth}
          \begin{description}
            \item[{FormationNum[4]}] 各机种数量
            \item[FormationHealth] 生命值
            \item[FormationFuel] 燃料
            \item[FormationAmmo] 弹药
            \item[FormationHealthMax] 生命值上限
            \item[{FormationSigntRange[3]}] 视野范围(三层)
            \item[{FormationAttack[2]}] 攻击力(火力伤害和鱼雷伤害)
            \item[FormationFuelMax] 燃料上限
            \item[FormationAmmoMax] 弹药上限
            \item[{FormationNumMax[4]}] 各机种初始数量
          \end{description}
        \end{minipage}
        \subparagraph{说明}
          由于编队的具体配置方案是在生产时由玩家自主决定的,故生命值上限、视野范围、攻击力、燃料上限、弹药上限、各机种初始数量的值只有在游戏过程中才能确定,除了初始数量,其他值还与当前生命值有关。其中视野范围只决定于编队是否有侦察机。

      \paragraph{常量}
        \begin{minipage}[t]{0.8\textwidth}
          \begin{description}
            \item[{FormationUnitAttack[4][3]}] 单位机种单位弹药攻击力
            \item[{FormationUnitHealthMax[4]}] 单位机种生命值上限
            \item[{FormationUnitFuelMax[4]}] 单位机种燃料上限
            \item[{FormationUnitAmmoMax[4]}] 单位机种弹药上限
            \item[ScoutFormationSightRange] 编队内有侦察机时的视野范围
            \item[NonScoutFormationSightRange] 编队内无侦察机时的视野范围
            \item[FormationSpeed] 单回合最大移动距离
          \end{description}
        \end{minipage}

      \paragraph{操作} 移动、攻击。
      \paragraph{其他说明}
        如果飞机编队受到攻击,则按侦察机、鱼雷机、轰炸机、战斗机的顺序依次扣除。


    \subsection{潜艇}
      \paragraph{状态量}
        \begin{minipage}[t]{0.8\textwidth}
          \begin{description}
            \item[SubmarineHealth] 生命值
            \item[SubmarineFuel] 燃料
            \item[SubmarineAmmo] 弹药
          \end{description}
        \end{minipage}

      \paragraph{常量}
        \begin{minipage}[t]{0.8\textwidth}
          \begin{description}
            \item[SubmarineHealthMax] 生命值上限
            \item[SubmarineFuelMax] 燃料上限
            \item[SubmarineAmmoMax] 弹药上限
            \item[{SubmarineSigntRange[3]}] 视野范围(三层)
            \item[{SubmarineFireRange[3]}] 攻击范围
            \item[{SubmarineAttack[2]}] 攻击力(火力伤害和鱼雷伤害)
            \item[SubmarineSpeed] 单回合最大移动距离
          \end{description}
        \end{minipage}

      \paragraph{操作} 移动、攻击。


    \subsection{潜艇}
      \paragraph{派生单位}
        驱逐舰(各方面能力均衡)、巡洋舰(航速大,机动力强)、航空母舰(航速小,兼备攻击和补给能力)、运输舰(无攻击能力,运输资源)
      \paragraph{状态量}
        \begin{minipage}[t]{0.8\textwidth}
          \begin{description}
            \item[ShipHealth] 生命值
            \item[ShipFuel] 燃料
            \item[ShipAmmo] 弹药
          \end{description}
        \end{minipage}

      \paragraph{常量}
        \begin{minipage}[t]{0.8\textwidth}
          \begin{description}
            \item[ ]  
          \end{description}
        \end{minipage}

      \paragraph{操作} 移动、攻击(除运输舰)、补给(运输舰和航母,只有航母可以补给飞机)、收集(运输舰独有功能,从岛屿、资源点收集资源运往其他位置)。


  \section{作战系统}
  \label{sec:作战系统}
      \paragraph{游戏模式}
        每回合双方玩家首先为所有可移动单位设定移动目的地 \ttname{Destination},如果未设定,则按上回合的目的地继续移动。再设定本回合移动后的指令 \ttname{Command}:攻击、补给、维修(每回合只能设定一条指令)。双方设定完所有指令后共同结算。

      \paragraph{胜利条件}
        摧毁对方基地 或 回合数达到最大回合数 \ttname{RoundMax}时,积分高的一方获胜。详见第\ref{sec:积分系统}节。

      \paragraph{距离}
        水平距离与竖直距离之和

      \paragraph{初始条件}
        双方所有单位(包括基地)资源充满,满血。岛屿满血无主。

      \paragraph{视野范围}
        地形全可见(海洋、基地、岛屿、油田、矿场),每个单位均有三层视野范围,己方单位共享视野范围,构成地图可见状态\mbox{ \ttfamily FieldofVison[MapSize][MapSize][3]},$-2$表示不可见,$-1$表示无单位,否则为单位ID。

      \paragraph{移动}
        移动时只能在同一层次内移动,且移动时不能穿越同层其他单位,同层单位亦不能堆叠于同一格。每移动一格消耗一单位燃料,燃料$\leq 0$则不能继续移动。特别地,若飞机编队燃料降至零,则坠毁,并对当前坐标的1层(水面)单位造成一定伤害。若飞机编队未发生实际移动(可能发出了移动指令但并未成功),则视为原地盘旋一回合,同样要消耗一单位燃料。

      \paragraph{特殊情况}
        当某两个单位在移动过程中将通过同一坐标时,则速度 \ttname{Speed}大的单位移动成功并继续移动,移动未成功的单位停留在前一格不能继续移动,若两单位速度相同,则采取随机的方式决定哪个单位移动成功。

      \paragraph{攻击}
        攻击在移动结束(包括上述意外停止的情况)后发动,攻击时需指定攻击的三维坐标点 \ttname{CommandCoordinate[3]},即攻击地图某一坐标的下中上三层的某一层。如果该坐标不在该单位的相应视野范围内,则返回错误值,视为放弃此次攻击。弹药量$\leq 0$时不能发动攻击。弹药在回合结束时到达该坐标产生伤害(即对方可以通过移动回避攻击)。异层攻击不考虑弹药攻击路径;某些情况(具体哪些情况暂未确定)需考虑弹药的攻击路径,若攻击路径上有其他单位,则弹药会击中障碍并产生相应伤害(注意:有友军伤害!)。

      \paragraph{补给}
        运输舰、航母、岛屿、基地特有指令,航母、岛屿、基地可以对任意单位补给,运输舰不能补给飞机。补给在移动结束后发动,同样需指定补给目标坐标,对空补给距离为0,对水面补给距离为1,对水下补给距离为0, 1。超出范围则视为放弃补给。每次补给补满对方容量,若剩余量不足以补满对方,则全部补充给对方。(这种情况下,运输舰、航母是否需要自动保留一些燃料供自己使用?)

      \paragraph{收集}
        运输舰特有指令,从岛屿或资源点获取资源,每次收集补满运输舰容量。收集距离为1。

      \paragraph{维修}
        基地特有指令,对空维修距离为0,其余维修距离为1,一次只能维修一个单位,维修补满生命值,补满弹药及燃料。若被维修单位为飞机,则类似于生产操作,可选择维修后的编队配置方案。


  \section{资源与生产系统}
    资源点分为油田和矿场,均在陆地地形上。

    \paragraph{油田}
      不可攻击,油储备总量 \ttname{TotalFuel},双方基地附近各有一小油田,地图的中间区域可设大油田,油田储备的燃料只能由运输舰收集获得(不可攻占,但可以派单位在周围保证运输舰运输安全)。

    \paragraph{矿产}
      不可攻击,金属储备总量 \ttname{TotalMetal},双方基地附近各有一小矿场,地图的中间区域可设大矿场,矿场储备的金属只能由运输舰收集获得(不可攻占,但可以派单位在周围保证运输舰运输安全)。

    \paragraph{生产 Build}
      基地除了常规指令外,额外拥有生产指令,给出各单位对应的编号,若生产飞机编队,还需给出编队配置方案~\ttname{BuildFormationNum[4]}。常量:生产飞机编队各机种所需资源及回合~\ttname{BuildFormation[4][4]}


  \section{积分系统}
  \label{sec:积分系统}
    积分 = 对敌军造成的伤害 + 占领岛屿(每回合加分)+ 收集的资源量


  \section{伤害计算}
    \subsection{制空权值}
      空对空伤害与制空权比值相关。
      \begin{equation}
        \text{制空权值} = \text{空对空伤害} + \text{距离为1的海军对空伤害}
      \end{equation}

    \subsection{伤害计算公式}
      \begin{equation}
        \text{伤害} = \text{攻击} – \text{防御}
      \end{equation}


    // 潜艇只能造成和接受鱼雷伤害。

    // 陆地建筑只能造成和接受火力伤害。

    // 飞机不能接受鱼雷伤害。


\end{document}
